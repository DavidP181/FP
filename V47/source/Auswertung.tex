\section{Auswetung}
\subsection{Messung der Molwärme bei konstantem Druck}
In Tabelle [1] sind die Messwerte der ersten Messreihe gelistet.
\begin{table}
  \centering
  \caption{Messergebnisse der ersten Messreihe der Probe und des äußeren Zylinders.}
  \label{tab1:w1}
  \begin{tabular}{
    S S S[table-format = 3.1]
    S[table-format = 2.2]
    S S[table-format = 1.1]
    }
    \toprule
    {R_P/\si{\ohm}} & {R_Z/\si{\ohm}} & {I_P/\si{\milli\ampere}} & {U_P/\si{\volt}} & {I_Z/\si{\ampere}} & {U_Z/\si{\volt}} \\
    \midrule
      22,7 &  22,9 &  151,2 &  15,80 & 3,0 & 1,7  \\
      26,4 &  24,7 &  152,6 &  15,80 & 3,0 & 1,7  \\
      29,5 &  26,7 &  153,2 &  15,80 & 3,0 & 2,0  \\
      32,5 &  29,2 &  153,6 &  15,80 & 3,6 & 2,6  \\
      35,0 &  32,4 &  153,8 &  15,80 & 4,6 & 3,8  \\
      37,8 &  36,8 &  153,9 &  15,80 & 4,6 & 5,0  \\
      40,7 &  41,1 &  154,1 &  15,80 & 4,0 & 5,1  \\
      43,6 &  45,5 &  154,2 &  15,80 & 3,4 & 5,0  \\
      46,9 &  49,8 &  154,3 &  15,80 & 3,4 & 5,1  \\
      49,9 &  53,8 &  154,4 &  15,80 & 3,2 & 5,0  \\
      53,0 &  57,7 &  154,4 &  15,80 & 3,2 & 5,1  \\
      56,0 &  61,5 &  154,5 &  15,80 & 3,2 & 5,1  \\
      59,0 &  65,3 &  154,5 &  15,80 & 3,0 & 5,1  \\
      62,1 &  69,1 &  154,6 &  16,33 & 3,0 & 5,1  \\
      65,1 &  72,8 &  154,6 &  16,34 & 3,0 & 5,1  \\
      68,1 &  76,6 &  154,6 &  16,34 & 2,6 & 4,5  \\
      71,2 &  80,4 &  154,6 &  16,35 & 2,4 & 4,0  \\
      74,1 &  83,1 &  154,7 &  16,35 & 2,4 & 4,0  \\
      77,0 &  84,0 &  154,7 &  16,35 & 2,4 & 4,0  \\
      79,8 &  86,0 &  154,7 &  16,35 & 3,0 & 5,0  \\
      83,1 &  88,2 &  154,7 &  16,35 & 3,0 & 5,0  \\
      86,3 &  90,4 &  154,7 &  16,35 & 3,0 & 5,0  \\
      89,3 &  92,6 &  154,7 &  16,35 & 3,0 & 5,0  \\
      92,3 &  95,0 &  154,7 &  16,34 & 3,0 & 5,0  \\
      95,3 &  97,6 &  154,8 &  16,35 & 2,8 & 5,0  \\
      97,3 &  99,4 &  154,8 &  16,34 & 2,8 & 5,1  \\
     101,1 & 101,1 &  154,8 &  16,34 & 2,8 & 5,0  \\
     102,9 & 102,9 &  154,8 &  16,34 & 2,8 & 5,0  \\
     104,5 & 104,5 &  154,8 &  16,34 & 3,0 & 5,5  \\
     107,7 & 107,7 &  177,4 &  18,71 & 3,2 & 6,0  \\
     109,9 & 109,9 &  177,4 &  18,71 & 3,0 & 6,0  \\
     112,5 & 112,5 &  177,5 &  18,71 & 3,2 & 6,5  \\
   bottomrule

  \end{tabular}

\end{table}

Es ist zu erkennen, dass beide Widerstände mit der Zeit und höherer Heizleistung
stetig ansteigen. Dabei weicht der Widerstand des Zylinders teilweise stark von dem
Widerstand der Probe ab.
Tabelle [2] zeigt die, aus den Widerständen berechneten Temperaturen an. Diese haben teilweise
Abweichungen von über $\SI{10}{\kelvin}$.

Aus diesen Werten lassen sich mit Gleichung () und () die jeweiligen Molwärmen mit
konstantem Druck bzw. Volumen berechnen. Die Ergebnisse sind in Tabelle [3] zu sehen.
Die Werte bei konstantem Druck unterscheiden sich kaum von den Werten mit konstantem Volumen.
Allerdings sind die Fehlerbereiche bei ca. einem Drittel der berechneten Werte.
Die berechnete Molwärme bleibt zunächst zwischen $\SI{15}{\joule\per\kelvin\mol} - \SI{20}{\joule\per\kelvin\mol}$, steigt aber
ab $\SI{270}{\kelvin}$ stark an.

Der theoretische Grenzwert der Molwärme liegt bei
\begin{equation}
  C_V = 3R = \SI{24,942}{\joule\per\kelvin\mol}.
\end{equation}
Dieser wird ab $\SI{266,7}{\kelvin}$ überschritten.


\subsection{Bestimmung der Debyetemperatur}

\begin{table}
  \centering
  \caption{Messwerte der zweiten Messreihe}
  \label{tab5:w2}
  \begin{tabular}{
    S[table-format = 2.1]
    @{${}\pm{}$}
    S[table-format=1.1]
    S[table-format = 2.1]
    @{${}\pm{}$}
    S[table-format=1.1]
    S[table-format = 3.1]
    @{${}\pm{}$}
    S[table-format=1.1]
    S[table-format = 2.2]
    @{${}\pm{}$}
    S[table-format=1.3]
    c c
    }
    \toprule
    {$R_P/\si{\ohm}$} & {$\sigma_Rp$} & {$R_Z/\si{\ohm}$} & {$\sigma_Rz$} & {$I_P/\si{\milli\ampere}$} & {$\sigma_Ip$} & {$U_P/\si{\volt}$} &{$\sigma_Up$} & {$I_Z/\si{\ampere}$} & {$U_Z/\si{\volt}$} \\
    \midrule
    22,4 & 0,1 & 22,4 & 0,1 & 163,2 & 0,5 & 17,09 & 0,05 & 3,4 & 1,5 \\
    25,9 & 0,1 & 24,0 & 0,1 & 164,3 & 0,5 & 17,22 & 0,05 & 3,4 & 1,5 \\
    28,3 & 0,1 & 25,5 & 0,1 & 164,8 & 0,5 & 17,30 & 0,05 & 3,6 & 2,0 \\
    30,0 & 0,1 & 26,6 & 0,1 & 165,1 & 0,5 & 17,35 & 0,05 & 4,0 & 2,0 \\
    31,4 & 0,1 & 27,6 & 0,1 & 165,3 & 0,5 & 17,37 & 0,05 & 4,0 & 2,0 \\
    32,9 & 0,1 & 30,0 & 0,1 & 165,4 & 0,5 & 17,39 & 0,05 & 5,5 & 3,0 \\
    34,9 & 0,1 & 32,9 & 0,1 & 165,5 & 0,5 & 17,42 & 0,05 & 6,0 & 3,5 \\
    37,0 & 0,1 & 32,9 & 0,1 & 165,6 & 0,5 & 17,43 & 0,05 & 6,0 & 3,5 \\
    39,0 & 0,1 & 35,0 & 0,1 & 165,7 & 0,5 & 17,45 & 0,05 & 6,5 & 4,0 \\
    41,2 & 0,1 & 37,1 & 0,1 & 165,8 & 0,5 & 17,47 & 0,05 & 6,5 & 4,5 \\
    43,5 & 0,1 & 39,7 & 0,1 & 165,9 & 0,5 & 17,49 & 0,05 & 6,0 & 5,0 \\
    45,3 & 0,1 & 42,3 & 0,1 & 165,9 & 0,5 & 17,50 & 0,05 & 7,0 & 5,0 \\
    47,7 & 0,1 & 47,3 & 0,1 & 166,0 & 0,5 & 17,51 & 0,05 & 7,0 & 5,5 \\
    50,2 & 0,1 & 50,1 & 0,1 & 166,0 & 0,5 & 17,53 & 0,05 & 7,0 & 6,0 \\
    53,0 & 0,1 & 52,6 & 0,1 & 166,1 & 0,5 & 17,54 & 0,05 & 7,0 & 6,0 \\
    55,7 & 0,1 & 55,4 & 0,1 & 166,2 & 0,5 & 17,55 & 0,05 & 7,0 & 6,5 \\
    58,1 & 0,1 & 57,9 & 0,1 & 166,2 & 0,5 & 17,56 & 0,05 & 7,0 & 7,0 \\
    60,5 & 0,1 & 60,4 & 0,1 & 166,3 & 0,5 & 17,57 & 0,05 & 7,0 & 7,0 \\
    63,0 & 0,1 & 63,2 & 0,1 & 166,3 & 0,5 & 17,58 & 0,05 & 7,0 & 7,5 \\
    65,6 & 0,1 & 65,6 & 0,1 & 166,3 & 0,5 & 17,59 & 0,05 & 7,0 & 7,5 \\
    \bottomrule

  \end{tabular}

\end{table}

In Tabelle [] ist eine zweite Messreihe aufgeführt, aus der die Wärmekapazität $C_V$ erneut bestimmt wird.
Die Debye-Temperatur wird mithilfe der Tabelle 1 aus der Anleitung bestimmt.
Dabei werden nur Messwerte betrachtet, bei denen die Temperatur der Probe kleiner als $170 \si{\kelvin}$ ist.
Die Funktion $C_V(\frac{\vartheta_D}{T})$ wird zwischen den diskreten Werten für $\frac{\vartheta_D}{T}$ als linear angenommen.
Aus der Interpolationsformel
\begin{align}
C_V - C_{V größer} &= \left( frac{\vartheta_D}{T} - \frac{\vartheta_{D größer}}{T} \right) \, \frac{C_{V kleiner} - C_V}{\vartheta_{D kleiner}- \vartheta_{D größer}} //
\frac{\vartheta_D}{T} = \frac{\vartheta_{D größer}}{T} + \left( C_V - C_{V größer} \right) \frac{\vartheta_{D kleiner}}{T} - \frac{\vartheta_{D größer}}{T}{C_{V kleiner}- C_{V größer}}
\end{align}
kann die Größe $\frac{\vartheta_D}{T}$ aus den Messwerten bestimmt werden.
Hierbei ist $C_{V größer}$ ausgehend vom Messwert der nächst größere Wert für $C_V$ aus der Tabelle und $C_{V kleiner}$ entsprechend der
nächst kleinere Wert.
Weiterhin ist $\frac{\vartheta_{D größer}}{T}$ der zugehörige Wert zu $C_{V größer}$ und $\frac{\vartheta_{D kleiner}}{T}$ der Wert zu $C_{V kleiner}$.
In Tabelle \ref{tab:Messreihe2} sind alle berechneten Größen aufgeführt.
\begin{table}
  \centering
  \caption{Wärmekapazität bei konstantem Volumen in Abhängigkeit zur Temperatur für die zweite Messreihe.}
  \label{tab:Messreihe2}
\begin{tabular}{
  S[table-format=3.2]
  @{${}\pm{}$}
  S[table-format=1.2]
  S[table-format=3.2]
  @{${}\pm{}$}
  S[table-format=1.2]
  S[table-format=2.1]
  @{${}\pm{}$}
  S[table-format=2.1]
  S[table-format=2.2] %29
  S[table-format=2.2]
  S[table-format=1.1] %30
  S[table-format=1.1] %30


  S[table-format=3.1]
  @{${}\pm{}$}
  S[table-format=4.1]
  }


    \toprule
    {$T_{Probe} \si{\per\kelvin}$} & {$\sigma_{T_Probe} \si{\per\kelvin}$}
    & {$T_{Rohr} \si{\per\kelvin}$} &{$\sigma_{T_Rohr} \si{\per\kelvin}$}
    & {$C_v \si{\joule\per\kelvin\mol}$} & {$\sigma_Cv \si{\joule\per\kelvin\mol}$}
    & {$C_{v kleiner} \si{\joule\per\kelvin\mol}$} & {$C_{v größer} \si{\joule\per\kelvin\mol}$}
    & {$\frac{\vartheta_D}{T}_{kleiner} $} & {$\frac{\vartheta_D}{T}_{größer} $}
    & {$\frac{\vartheta_D}{T} $} & {$\frac{\vartheta_D}{T} $}
    & {$\vartheta_D \si{\per\kelvin} $} & {$\sigma_{\vartheta_D} \si{\per\kelvin} $} //
    \midrule
       82.23  & 0.24 & 82.08  & 0.24 & 18.8 & 5.8 & 18.60 & 19.01 & 2.5 & 2.6 & 2.5 & 5.0  & 210 & 200 \\
       90.50  & 0.24 & 85.86  & 0.26 & 27.7 & 8.6 & 24.94 & ---  & 0.0 & --- \\
       96.18  & 0.23 & 89.40  & 0.24 & 39.4 & 12.4 & 24.94 & --- & 0.0 & ---  \\
       100.22 & 0.24 & 92.00  & 0.24 & 48.0 & 15.3 & 24.94 & --- & 0.0 & ---  \\
       103.55 & 0.24 & 94.37  & 2.24 & 44.8 & 14.2 & 24.94 & --- & 0.0 & ---  \\
       107.12 & 0.24 & 100.07 & 0.24 & 33.6 & 10.4 & 24.94 & --- & 0.0 & ---  \\
       111.89 & 0.24 & 106.97 & 0.24 & 32.0 & 10.0 & 24.94 & --- & 0.0 & ---  \\
       116.92 & 0.24 & 106.97 & 0.24 & 33.6 & 10.4 & 24.94 & --- & 0.0 & ---  \\
       121.71 & 0.24 & 111.98 & 0.24 & 30.5 & 9.4  & 24.94 & --- & 0.0 & ---  \\
       127.00 & 0.24 & 117.01 & 0.24 & 29.1 & 9.0  & 24.94 & --- & 0.0 & ---  \\
       132.54 & 0.24 & 123.24 & 0.24 & 37.2 & 11.6 & 24.94 & --- & 0.0 & ---  \\
       136.89 & 0.24 & 129.50 & 0.24 & 27.9 & 8.6  & 24.94 & --- & 0.0 & ---   \\
       142.70 & 0.24 & 141.58 & 0.24 & 26.7 & 8.2  & 24.94 & --- & 0.0 & ---  \\
       148.77 & 0.24 & 148.37 & 0.24 & 23.8 & 7.3 & 23.74 & 23.96 & 1.0 & 1.1 & 10 & 45 & 100 & 6700 \\
       155.58 & 0.24 & 154.46 & 0.24 & 24.6 & 7.5 & 24.50 & 24.63 & 0.5 & 0.6 & 0.6 & 5.3 & 90 & 830 \\
       162.17 & 0.24 & 161.29 & 0.24 & 27.7 & 8.5  & 24.94 & --- & 0.0 & ---  \\
       168.05 & 0.25 & 167.41 & 0.25 & 27.6 & 8.5  & 24.94 & --- & 0.0 & ---  \\
    \bottomrule
  \end{tabular}
\end{table}

In der letzten Spalte ist das $\vartheta_D$ explizit berechnet.
Die gewichtete Mittelung ergibt den Wert
\begin{align}
  \vartheta_{D mittel} = 203.33 \, \si{\Kelvin}
\end{align}
Hierfür wird die Formel
\begin{align}
    \vartheta_{D mittel} &= \frac{\sum_{\mathclap{i}}  \vartheta_{D mittel} g_i}{\sum_{\mathclap{i}} g_i}
g_i &= \left( \increment \vartheta_{D i} \right) ^2
\end{align}
verwendet.

\subsection{Theoretische berechnung von $\omega_D$ und $\theta_D$}
Aus Gleichung () lässt sich $\omega_D$ berechnen. Dazu wurden
$v_l = \SI{4,7}{\kilo\meter\per\second}$, $v_t = \SI{2,26}{\kilo\meter\per\second}$
und $N=N_A \cdot \frac{m}{M}$ eingesetzt. Es ergibt sich
\begin{equation}
  \omega_D = \SI{4,3463 \cdot 10^{13}}{\per\second}.
\end{equation}

Daraus lässt sich $\theta_D$ bestimmen.
\begin{equation}
  \theta_D = \frac{\hbar\omega_D}{k_B} = \SI{331,98}{\kelvin}
\end{equation}
