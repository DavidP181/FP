\section{Auswetung}
\subsection{Messung der Molwärme bei konstantem Druck}
In Tabelle [1] sind die Messwerte der ersten Messreihe gelistet.
\begin{table}
  \centering
  \caption{Messergebnisse der ersten Messreihe der Probe und des äußeren Zylinders.}
  \label{tab1:w1}
  \begin{tabular}{
    S S S[table-format = 3.1]
    S[table-format = 2.2]
    S S[table-format = 1.1]
    }
    \toprule
    {R_P/\si{\ohm}} & {R_Z/\si{\ohm}} & {I_P/\si{\milli\ampere}} & {U_P/\si{\volt}} & {I_Z/\si{\ampere}} & {U_Z/\si{\volt}} \\
    \midrule
      22,7 &  22,9 &  151,2 &  15,80 & 3,0 & 1,7  \\
      26,4 &  24,7 &  152,6 &  15,80 & 3,0 & 1,7  \\
      29,5 &  26,7 &  153,2 &  15,80 & 3,0 & 2,0  \\
      32,5 &  29,2 &  153,6 &  15,80 & 3,6 & 2,6  \\
      35,0 &  32,4 &  153,8 &  15,80 & 4,6 & 3,8  \\
      37,8 &  36,8 &  153,9 &  15,80 & 4,6 & 5,0  \\
      40,7 &  41,1 &  154,1 &  15,80 & 4,0 & 5,1  \\
      43,6 &  45,5 &  154,2 &  15,80 & 3,4 & 5,0  \\
      46,9 &  49,8 &  154,3 &  15,80 & 3,4 & 5,1  \\
      49,9 &  53,8 &  154,4 &  15,80 & 3,2 & 5,0  \\
      53,0 &  57,7 &  154,4 &  15,80 & 3,2 & 5,1  \\
      56,0 &  61,5 &  154,5 &  15,80 & 3,2 & 5,1  \\
      59,0 &  65,3 &  154,5 &  15,80 & 3,0 & 5,1  \\
      62,1 &  69,1 &  154,6 &  16,33 & 3,0 & 5,1  \\
      65,1 &  72,8 &  154,6 &  16,34 & 3,0 & 5,1  \\
      68,1 &  76,6 &  154,6 &  16,34 & 2,6 & 4,5  \\
      71,2 &  80,4 &  154,6 &  16,35 & 2,4 & 4,0  \\
      74,1 &  83,1 &  154,7 &  16,35 & 2,4 & 4,0  \\
      77,0 &  84,0 &  154,7 &  16,35 & 2,4 & 4,0  \\
      79,8 &  86,0 &  154,7 &  16,35 & 3,0 & 5,0  \\
      83,1 &  88,2 &  154,7 &  16,35 & 3,0 & 5,0  \\
      86,3 &  90,4 &  154,7 &  16,35 & 3,0 & 5,0  \\
      89,3 &  92,6 &  154,7 &  16,35 & 3,0 & 5,0  \\
      92,3 &  95,0 &  154,7 &  16,34 & 3,0 & 5,0  \\
      95,3 &  97,6 &  154,8 &  16,35 & 2,8 & 5,0  \\
      97,3 &  99,4 &  154,8 &  16,34 & 2,8 & 5,1  \\
     101,1 & 101,1 &  154,8 &  16,34 & 2,8 & 5,0  \\
     102,9 & 102,9 &  154,8 &  16,34 & 2,8 & 5,0  \\
     104,5 & 104,5 &  154,8 &  16,34 & 3,0 & 5,5  \\
     107,7 & 107,7 &  177,4 &  18,71 & 3,2 & 6,0  \\
     109,9 & 109,9 &  177,4 &  18,71 & 3,0 & 6,0  \\
     112,5 & 112,5 &  177,5 &  18,71 & 3,2 & 6,5  \\
   bottomrule

  \end{tabular}

\end{table}

Es ist zu erkennen, dass beide Widerstände mit der Zeit und höherer Heizleistung
stetig ansteigen. Dabei weicht der Widerstand des Zylinders teilweise stark von dem
Widerstand der Probe ab.
Aus diesen Werten lassen sich mit Gleichung () und () die jeweiligen Molwärmen mit
konstantem Druck bzw. Volumen berechnen. Die Ergebnisse sind in Tabelle [2] zu sehen.
